\documentclass[12pt]{article} 
\usepackage[xetex, a4paper, left=2cm, right=2cm, top=2cm,bottom=2cm]{geometry}
\usepackage[cm-default]{fontspec}
\usepackage{xunicode}

%\tolerance=1000
%\emergencystretch=0.74cm 

\usepackage{polyglossia}
\setdefaultlanguage[spelling=modern]{russian}
\setotherlanguage{english} 
\defaultfontfeatures{Scale=MatchLowercase,Ligatures=TeX}  %% устанавливает поведение шрифтов по умолчанию  
\newfontfamily\cyrillicfont{XITS} 
\setromanfont[Mapping=tex-text]{XITS}
\setsansfont[Mapping=tex-text]{XITS}
\setmonofont{XITS}
%\newfontfamily\cyrillicfont{Liberation Mono} 

%\usepackage{makecell}

%\usepackage{titlesec}
%\newcommand{\sectionbreak}{\clearpage}

%\renewcommand{\thesection}{\Alph{section}}
%\newcount\wd    \wd=\textwidth \multiply\wd by 8 \divide\wd by 17

\usepackage[unicode, pdfborder={0 0 0 0}]{hyperref}

\usepackage{minted}
\usemintedstyle{friendly}

\author{Andrei Pazniak}
\title{Описание библиотеки времени исполнения Phobos}

%желательно добавить примеров к каждому описанию
\begin{document}
\hypersetup{
pdftitle = {Phobos description},
pdfauthor = {Andrei Pazniak},
pdfsubject = {Описание библиотеки времени исполнения Phobos}
}% End of hypersetup

\section{std.algorithm}
Многие функции, описанные в модуле, параметризованны предикатом\footnote{в качестве предиката можно указать имя функции, 
имя делегата, имя функтора или строку с программой на D, которая может быть скомпилирована}. Предикат может содержать любые 
допустимые выражения языка D, которые используют символ 'a' (для унарных функций) или символы 'a' и 'b' 
(для бинарных функций). Гарантируется, что эти имена не вызовут конфликта с такими же именами в пользовательском коде, 
так как обрабатываются в другом контексте.
\begin{description}
\item [all] \hfill \\
\begin{minted}{d}
template all(alias pred = "a")
bool all(Range)(Range range) 
        if (isInputRange!Range && is(typeof(unaryFun!pred(range.front))));
\end{minted}

Проверяет, все ли элементы диапазона\footnote{Range} range удовлетворяют предикату pred. 
Можно явно не указывать предикат, если элемнты диапазона приводятся к булевому типу (true, false). Это позволяет легко проверить,
все ли элементы диапазона принимают значение true. В худшем случае выполняет range.length проверок.
\item [any] \hfill \\
\begin{minted}{d}
template any(alias pred = "a")
bool any(Range)(Range range) 
        if (isInputRange!Range && is(typeof(unaryFun!pred(range.front))));
\end{minted}

Проверяет, существует ли элемент диапазона range, который удовлетворяет предикату pred. 
Можно явно не указывать предикат, если элемнты диапазона приводятся к булевому типу (true, false). Это позволяет легко проверить,
существует ли элемент диапазона, который принимает значение true. В худшем случае выполняет range.length проверок.
\item [balancedParens] \hfill \\
\begin{minted}{d}
bool balancedParens(Range, E)(Range r, E lPar, E rPar, size_t maxNestingLevel = size_t.max)
        if (isInputRange!Range && is(typeof(r.front == lPar)));
\end{minted}
Проверяет, "баланс скобок" в диапазоне r, то есть равно ли количество "открывающих" 
элементов lPar количеству "закрывающих" rPar. Параметр maxNestingLevel указывает максимально допущенный уровень вложенности
скобок. maxNestingLevel = 0 говорит о том, что не допускаются вложенные скобки.

\item [canFind] \hfill \\
\begin{minted}{d}
template canFind(alias pred = "a == b")
\end{minted}
Удобная функция, которая проверяет, успешно ли выполнится поиск.
\begin{minted}{d}
bool canFind(Range)(Range haystack) 
        if (is(typeof(find!pred(haystack))));
\end{minted}
Возвращает true в том случае, когда для любого элемента v в диапазоне haystack выполняется предикат, иначе false. 
В Худшем случае выполняет haystack.length проверок предиката.
\begin{minted}{d}
bool canFind(Range, Element)(Range haystack, Element needle) 
        if (is(typeof(find!pred(haystack, needle))));
\end{minted}
Возвращает true тогда и только тогда, когда needle может быть найдена в haystack, иначе возвращает false.
В худшем случае выполняет haystack.length проверок предиката.
%По-хорошему нужно дописать описание 3-ей функции, но я её не понимаю

\item [count] \hfill \\
\begin{minted}{d}
size_t count(alias pred = "a == b", Range, E)(Range haystack, E needle) 
        if (isInputRange!Range && !isInfinite!Range && is(typeof(binaryFun!pred(haystack.front, needle)) : bool)); 
\end{minted}
Подсчитывает количество таких элементов v диапазона r, для которых pred(v, needle) == true. По умолчанию они сравниваются на равенство.
В худшем случае выполняется haystack.length проверок предиката.
\begin{minted}{d}
size_t count(alias pred = "a == b", R1, R2)(R1 haystack, R2 needle) 
        if (isForwardRange!R1 && !isInfinite!R1 && isForwardRange!R2 && is(typeof(binaryFun!pred(haystack.front, needle.front)) : bool)); 
\end{minted}
Подсчитывает, сколько раз в haystack встречается needle. Порождает исключение, если needle.empty, так как пустой диапазон встречается
бесконечное число раз в любом диапазоне. Не рассматривается возможным наложение искомых элементов друг на друга. Например
count ("aaa", "aa") равняется 1, а не 2.
\begin{minted}{d}
size_t count(alias pred = "true", R)(R haystack) 
        if (isInputRange!R && !isInfinite!R && is(typeof(unaryFun!pred(haystack.front)) : bool));
\end{minted}
Подсчитывает количество элементов, для которых выполняется предикат. В худшем случае выполняет haystack.length проверок предиката.


\end{description}
\end{document}